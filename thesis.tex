\documentclass[a4paper]{book}


% --- Package --- %
\usepackage[a4paper]{meta-donnees}     % pour mettre la page de garde
\usepackage{subfiles}                  % pour subdiviser les fichiers .tex
\usepackage[english]{babel}            % pour transformer chapter -> chapitre\\
\selectlanguage{english}
\usepackage[utf8]{inputenc}            % pour mettre les accents
\usepackage[fontsize=12pt]{scrextend}  % pour changer la taille de la police
\usepackage[inner  = 3.5cm,
            outer  = 2.7cm,
            top    = 3.5cm,
            bottom = 3.5cm]{geometry}  % pour les marges changeantes interieur et extérieur        
\usepackage{xcolor,lipsum}             % Required for specifying colors by name
%\usepackage{avant}                     % Use the Avantgarde font for headings
\usepackage{mathptmx}                  % Use the Adobe Times Roman as the default text font together with math symbols
\usepackage{microtype}                 % Slightly tweak font spacing for aesthetics
\usepackage[T1]{fontenc}               % Use 8-bit encoding that has 256 glyphs
\usepackage{ctable}                    % for toprule, midrule, bottomrule
\usepackage{titlesec}                  % for titleformat
\usepackage{multicol}                  % for double colunm items
\usepackage{amsmath}                   % for math formula
\usepackage{amssymb}                   % for math formula
\usepackage{placeins}                  % for floatBarrier
\usepackage{lmodern}                   % for the front
\usepackage{graphicx}                  % for includegraphics
\usepackage{array}                     % for multicolumn table
\usepackage{multirow}                  % use of multirow
\usepackage{fancyhdr}                  % to use the fancy style
\usepackage{lipsum}                    % for dummy test
%\pagestyle{fancy}                      % for fancy style
\usepackage{titlesec}                  % for the title section (mise en page chap.)
\usepackage{lettrine}                  % For the drop cap
\usepackage{multicol}
\usepackage{chngpage}

\newenvironment{custommargins}[2]{
     \addtolength{\leftskip}{#1}
     \addtolength{\rightskip}{#2}}{\par}

%\usepackage{lastpage}                 % Used to determine the number of pages in the document (for "Page X of Total")* 
%\usepackage{bm}
\usepackage[nottoc, notlof, notlot]{tocbibind}
\renewcommand{\baselinestretch}{1.2}
\renewcommand{\thefootnote}{\alph{footnote}}

\usepackage[noadjust]{cite}
\renewcommand{\citedash}{--} 
\usepackage{rotating}
% --------------- %

\usepackage{hyperref}
\hypersetup{
    colorlinks,
    citecolor=black,
    filecolor=black,
    linkcolor=black,
    urlcolor=black
}

% --- Style --- %
\pagestyle{fancy}
\fancyhf{}
\fancyhead[LE,RO]{\leftmark}
%\fancyhead[RE,LO]{\leftmark}
%\fancyfoot[CE,CO]{\leftmark}
\fancyfoot[LE,RO]{\thepage}
% ------------- %
 
 
% --- New command --- %
\renewcommand {\headrulewidth} {2pt}          % Change l'épaisseur du trait du haut de page
\renewcommand {\footrulewidth} {1.5pt}        % Change l'épaisseur du trait du pied de page
\renewcommand {\familydefault} {\sfdefault}   % Change la police des haut et pied de page
\newcommand*  {\horzbar}       {\rule[2pt]{2pt}{2pt}}
\newcommand{\Cd}{$^{\text{116}}$Cd}
\newcommand{\Sn}{$^{\text{116}}$Sn}
\newcommand{\NI}{\noindent}
% ------------------- %


% --- Change la mise en page des chapitre --- %
\usepackage[a]{fncychap}
\makeatletter
	\ChTitleVar{\Huge\sffamily}
	\ChNameVar{\bfseries\LARGE\sffamily}
\makeatother
% ------------------------------------------ %




\newenvironment{changemargin}[2]{\begin{list}{}{%
\setlength{\topsep}{0pt}%
\setlength{\topmargin}{2pt}%
\setlength{\leftmargin}{0pt}%
\setlength{\rightmargin}{0pt}%
\setlength{\listparindent}{\parindent}%
\setlength{\itemindent}{\parindent}%
\setlength{\parsep}{0pt plus 1pt}%
\addtolength{\leftmargin}{#1}%
\addtolength{\rightmargin}{#2}%
}\item }{\end{list}}

  
\begin{document}


% --- Option pour les espaces --- %
%\Sethpageshift{???mm}   %%optionnel : à décommenter si besoin pour ajout d'espace afin de center la couvérture horizontalement (valeur par défaut est -5.5mm)
%\Setvpageshift{???mm}   %%optionnel : à décommenter si besoin pour ajout d'espace afin de center la couvérture verticalement (valeur par défaut est -15.5mm)

%\Universite{}    %%optionnel : à décommenter et à renseigenr si vous voulez changer le non d'université
%\Grade{}         %%optionnel : à décommenter et à renseigenr si vous voulez changer le grade
% ------------------------------- %


% --- Page de garde --- %
\Specialite      {Physique Subatomique et Astroparticules}
\Arrete          {25 mai 2016} % ATTENTION, A VERIFIER CHAQUE ANNEE
\Auteur          {Thibaud LE NOBLET}
\Directeur       {Dominique DUCHESNEAU}
\CoDirecteur     {Alberto REMOTO}
\Laboratoire     {Laboratoire d'Annecy-le-Vieux de Physique des Particules}
\EcoleDoctorale  {Communaut\'e Universit\'e Grenoble Alpes}
\Titre           {Background studies and design optimisation of the SuperNEMO demonstrator module\\
\smallskip
Search for $2\nu\beta\beta$ and $0\nu\beta\beta$ decays of $^{\text{116}}$Cd into the excited states of $^{\text{116}}$Sn with NEMO-3}

%\Soustitre{}%%optionnel : à décommenter et à renseigenr si présence d'un sous-titre de thèse
\Depot           {date}

% --- Commande pour création de nouvelles catégories dans le jury --- %
%\UGTNewJuryCategory{...NomDeLaCategorie...}{...Definition...}
% Exemple \UGTNewJuryCategory{UGTFamille}{Membre de la famille} que nous ajoutons dans la commande \Jury ci-dessous sous la forme \UGTFamille{Jean Rousseau}{(...titre_et_affiliation...s'il_y_en_a...)}
% ------------------------------------------------------------------- %


\Jury  {%\UGTRole{civilité, prénom et nom}{titre}{affiliation}
  \UGTPresidente    {Mme, M ??}{??}{}
  \UGTRapporteur    {M, Claudio GIGANTI}    {Chargé de recherche, LPNHE - Paris}             {}
  \UGTRapporteur    {M, Ruben SAAKYAN}      {Professor, UCL - Londres}                       {}
  \UGTExaminatrice  {Mme, Jaime DAWSON}     {Chargée de recherche, APC - Paris}              {}
  \UGTExaminatrice  {Mme, Edwige TOURNEFIER}{Directrice de recherche, LAPP - Annecy}         {}
  \UGTExaminateur   {M, Laurent SIMARD}     {Ma\^itre de conf, Univ. Paris-Sud / LAL - Orsay}{}
  %\UGTDirecteur     {M, le Directeur}{DR}{Labo1}
}
\MakeUGthesePDG    %% très important pour générer la couvérture de thèse

\tableofcontents
\addcontentsline{toc}{chapter}{Introduction}

% --- Parties de la thèse --- %
\subfile{Introduction}

\subfile{Chap1-NeutrinoPhysics}

\subfile{Chap2-DoubleBeta}

\subfile{Chap3-NEMOdetectors}

\subfile{Chap4-SourceFoilDesign}

\subfile{Chap5-RadonBkgStudy}

\subfile{Chap6-Cd116ExcitedStatesAnalysis}

\subfile{Conclusion}


\appendix

\subfile{Annexe1-SourceFabrication}

\subfile{Annexe2-LightInjection}




\bibliographystyle{ieeetr}
\bibliography{biblio}

\newpage
\cleardoublepage

\thispagestyle{empty}

\subfile{Summary}

\end{document}
