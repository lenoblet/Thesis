\documentclass[main.tex]{subfiles}

\setlength{\topmargin}{1cm}
\setlength{\leftmargin}{1cm}
\setlength{\columnsep}{1.5cm}

\begin{document}

%\begin{changemargin}{-2cm}{-1cm}

%\begin{multicols}{2}


\NI \textbf{R\'esum\'e}

\bigskip

\NI \small{Les détecteurs NEMO-3 et SuperNEMO ont été conçus pour la recherche de décroissance double bêta sans émission de neutrinos. Ces détecteurs fournissent une approche unique dans la recherche des événements double bêta en combinant des mesures à la fois de trajectoires, de temps de vol et d'énergie. De plus, grâce à la séparation de la source double bêta du reste du système de détection, les détecteurs NEMO ont la possibilité d'étudier plusieurs isotopes double bêta avec une forte réjection du bruit de fond. \\
\noindent Cette thèse présente plusieurs études réalisées dans le cadre de l'optimisation et la préparation du détecteur SuperNEMO, successeur de NEMO-3. La première concerne l'optimisation des performances du détecteur en fonction de la configuration mécanique de ses feuilles sources. La conclusion de cette étude est que les deux configurations considérées sont équivalentes. La seconde étude s'intéresse à l'un des principaux bruits de fond que constitue le radon dans la recherche des désintégrations double bêta. Cette étude a été concrétisée par le développement et l'implémentation d'un algorithme permettant l'identification et la mesure des événements provenant de ce bruit de fond. Le deuxième volet de cette thèse rapporte l'analyse des données de NEMO-3 pour rechercher les décroissances double bêta avec et sans émission de neutrino du $^{\text{116}}$Cd vers les états excités du $^{\text{116}}$Sn. Ces décroissances n'ont jamais été observées à ce jour, et les limites obtenus sur les états excités (2$^+$) et (0$^+$) sont les premières utilisant le détecteur NEMO-3.}


\bigskip

\NI \textbf{Mots clés}

\smallskip

\NI \small{Neutrino, Double bêta, SuperNEMO, Feuille source, Radon, État excité, Cadmium-116}


\vspace{3cm}


\NI \textbf{Abstract}

\bigskip

\noindent The NEMO-3 and SuperNEMO detectors have been designed to search for neutrinoless double beta decays. These detectors provide a unique approach combining a calorimetric and a tracking measurement of double beta events emitted by a separated isotopic source. This approach allows to search for neutrinoless double beta decays among several isotopes with good background rejection. \\
\noindent This thesis presents many studies performed for the optimisation and the preparation of the SuperNEMO detector, successor of NEMO-3. The first study concerns the optimisation of the detector performances with respect to the design of the source foil. The conclusion of this study is that the two configurations considered are equivalent. The second study focuses on the radon which constitutes one of the main background to the search for double beta decays. In this study an algorithm has been developed and implemented to search for the alpha particle allowing the identification and the measurement of the radon events. The thesis is completed by an analysis of the NEMO-3 data to search for the double beta decay of $^{\text{116}}$Cd via the excited state of $^{\text{116}}$Sn. These decays have never been observed up to date and the limits set on the excited states (2$^+$) and (0$^+$) are the first using the NEMO-3 detector.

\NI \textbf{Key Words}

\smallskip

\NI \small{Neutrino, Double beta decay, SuperNEMO, Source Foil, Radon, Excited State, Cadmium-116}


%\end{multicols}

%\end{changemargin}

\end{document}
