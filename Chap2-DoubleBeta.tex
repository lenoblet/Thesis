\documentclass[main.tex]{subfiles}
\begin{document}

\chapter{Double beta decay}
 \begin{flushright}
\textit{blablab lablab labla labla labla labla labla lablalabla \\ 
labla lablala labla labla lablabla labla}\\
Author.
\end{flushright} 


\section{The radioactivity}
\NI parler des 3 types

\section{Two neutrino double beta decay}
\NI explication, c'est autorisé dans le MS

\subsection{Phenomenology}
\subsection{Different isotopes}

\section{Neutrinoless double beta decay}
\NI explications , c'est interdit dans le MS, plot meff, ne pas oublier de spécifier que c'est relier avec les paramètres d'oscillations du neutrino.
\subsection{Phenomenology}
\NI mettre la formule avec l'espace de phase, les éléments de matrices nucléaires et le éta. Dans le cas le plus simple, échange d'un neutrino léger de Majorana.

\subsection{Effective mass of Majorana}
\NI mettre ici le plot meff.

\subsection{Nuclear matrix elements}

\subsection{Other processes behond $\beta\beta0\nu$}
\NI Faire la liste des autres processus.

\section{Experimental point of view}
\NI mettre formule sens, il faut des détecteurs bas bruit de fond, ... radiopureté ... parler de l'efficacité,
\NI deux approches sont aujourd'hui considérés : tracko calo et juste calo. Point fort et point faible de chaque approche


\section{Status of double beta experiments}
\NI Faire un statut de la double beta aujourd'hui avec 
\NI Gerda, KamLandZen, Cuore.
\subsection{KamLAND-Zen}
\subsection{GERDA}
\subsection{Cuore}
\end{document}