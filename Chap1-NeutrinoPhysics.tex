\documentclass[main.tex]{subfiles}
\begin{document}

\chapter{Neutrino physics}
 \begin{flushright}
\textit{blablab lablab labla labla labla labla labla lablalabla \\ 
labla lablala labla labla lablabla labla}\\
Author.
\end{flushright} 


\section{Neutrino in Standard Model}

\subsection{The Standard Model}
\NI Parler du modèle standard, théorie décrivant les interactions entre les particules, parler des 4 forces, mettre un tableau des particules, parler un peu du Higgs.

\subsection{Leptons and weak interaction}
\NI ce sera une partie plus théorique.

\subsection{Neutrinos flavors}
les 3 saveurs, quand elles ont été découvertes et par quelle expérience. Dans le MS les neutrinos ont une masse nulle. (finir cette partie par une ouverture, "le modèle standard ne serait qu'une approximation à basse énergie d'une théorie plus globale., beaucoup de physiciens cherche aujourd'hui une preuve d'une physique au delà du MS.")

\section{Neutrino mixing and oscillations}
\subsection{Solar neutrino anomaly}
\NI faire un bref rappel

\subsection{Atmospheric neutrino anomaly}
\NI de m\^eme que la partie précédente.

\subsection{Phenomenology of neutrino oscillations}
\NI Pontecorvo, similaire à l'oscillation k0 K0bar dans le secteur des quarks, matrice PMNS.

\subsection{Observations and status}
\NI Dire qu'aujourd'hui on les a observées. On connait et on mesure de mieux en mieux les paramètres (angles de mélanges et différences de masse au carré. Faire un tableau récap). Mettre aussi ce que l'on ne connait pas. La conséquence directe de l'oscillation des neutrinos est qu'ils doivent \^etre massifs, on peut voir cela comme une première preuve d'une physique au delà du modèle standard. 


\section{Neutrino mass}
\subsection{Dirac term mass}
\subsection{Majorana term mass}
\subsection{See-saw mechanism}
\subsection{Neutrino mass measurements}
\NI Dire que depuis qu'on sait qu'ils sont massifs, on cherche à mesurer leur masse. C'est un challenge, d'autant plus que les expériences d'oscillations ne le permettent pas. Mesure directe, cosmo, double b\^eta. Mettre les dernières mesures.


\end{document}